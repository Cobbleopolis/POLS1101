\section{Georgia's Constitution}
Longer than federal constitution because:
\begin{enumerate}
    \item Lack of implied powers. All powers must be explicitly stated.
    \item Political reasons. Political parties or interest groups will try to amend the constitution to lock in their view on a topic. Example, Same-sex Marriage (Constitutional Amendment 1 (2004): Made it unconstitutional for the state to recognize same-sex marriages or civil unions).
    \item As a response to state court decisions. Almost the only way to undo a court's action is to amend the constitution. Example, Nude Dancing.
    \item National government requirements. Example, Medicaid. The Georgia Constitution was amended to create and trust fund to provide medical services for the poor.
\end{enumerate}
\subsection{Georgia Constitution vs. US Constitution}
\subsubsection{Both:}
\begin{enumerate}
    \item Have bill of rights.
    \item Adopt a separation of powers.
    \item Executives have the power to appoint officials and veto bills.
    \item Legislatures are bicameral.
\end{enumerate}
\subsubsection{Differences:}
\begin{enumerate}
    \item Georgia voters must approve all amendments.
    \item Georgia Constitution requires the state to have a balanced budget.
    \item The Georgia governor is granted the line-item veto (They have the ability to veto a specific portion of a bill rather than veto the bill in its entirety).
    \item All legislators serve 2-year terms.
    \item Representation is based on population (in both houses).
    \item Georgia has a ``plural executive'' (Georgia voters get to elect 6 main officials rather than being appointed).
    \item Almost all judges in Georgia are elected.
\end{enumerate}
Georgia's current constitution (1983) is $10^{th}$ in Georgia's history
Notable changes from rpevious constitution:
\begin{itemize}
    \item Local government proposals not required on state ballot
    \item Unified court system. Streamlined hierarchy of court levels.
    \item Nonpartisan election of state court judges
    \item Enhanced General Assembly powers, removed some powers from the governor and gave it to the General Assembly.
    \item Equal protection clause.
\end{itemize}
\subsection{Constitutional Amendments}
Georgia's constitution can be amended two ways:
\begin{itemize}
    \item The General Assembly can ask voters to create a convention to amend/replace this Constitution.
    \item The general Assembly can submit proposed amendments to voters by a 2/3 vote in each house. Voters have approved 76\% of proposed amendments since 1984.
\end{itemize}
\section{Georgia's Government}
\subsection{General Assembly}
\begin{itemize}
    \item Has 236 members ($3^{rd}$ largest).
    \item No limit on number of terms that one can serve.
    \item Can override regular and line-item vetos by 2/3 majority in each house.
    \item Has unlimited power to change the budget, which the governor must submit each year.
\end{itemize}
\subsection{The Governor}
\begin{itemize}
    \item Can server 2 consecutive, 4 year terms.
    \item Along with other state officials, they can be recalled.
    \item Has weak appointment powers. 
    \begin{itemize}
        \item Voters select many top administrators.
        \item But governors can fill vacancies.
    \end{itemize}
    \item Has veto and line-item veto powers.
    \begin{itemize}
        \item Doesn't have pocket veto. If it is ignored the bill \textit{will} become law.
    \end{itemize}
\end{itemize}
\subsection{Elections}
\subsubsection{Runoffs}
\begin{itemize}
    \item Used in Georgia when no single candidate is able to capture 50\% of the vote.
    \item Seen as biased against minority candidates, who might finish first in election but not get a majority.
    \begin{itemize}
        \item In the runoff, whites might vote in a bloc for the remaining white candidate.
    \end{itemize}
    Challenged at the supreme court level but ruled as legal
\end{itemize}