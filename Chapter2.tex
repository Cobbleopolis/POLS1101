\section{The Constitution}
\subsection{The Articles of Confederation}
The revolutionaries proclaimed the creation of a republic
\begin{itemize}
    \item A government without a monarch.
    \item Government based on the consent of the governed, whose power is exercised by elected representatives.
    \item Need not be a democracy.
\end{itemize}
Adopted by the Continental Congress on Nov. 15, 1777.
Jealously guarded state sovereignty.
Shaped in part by the Revolutionary experience.
\subsubsection{Reasons the Articles of Confederation}
\begin{itemize}
    \item Didn't give taxing power to the national government.
    \item Didn't provide for a strong executive.
    \item Didn't allow national government to regulate commerce.
    \item Amendments required unanimity.
    \item Ill-equipped to deal with domestic disorder.
    \begin{itemize}
        \item Example: Shay's Rebellion (1786-1787)
    \end{itemize}
\end{itemize}
\subsection{The Philadelphia Convention (1787)}
Took place after Shay's Rebellion because it showed that the government isn't fulfilling one of the basic goals of government, to provide order.

Intended to amend the Articles of Confederation.
\subsubsection{The Debate Over State Representation}
\begin{description}
    \item[The Virginia Plan] Sought to replace the weak confederation with a powerful national government.
    \item[The New Jersey Plan] Sought to preserve the Articles of Confederation.
\end{description}
The New Jersey Plan was favored by the small states. Feared that representation based solely on population would minimize their influence. Defeated 7-3.
\subsubsection{The Great Compromise}
The Legislature would be bicameral.

The House of Representatives -- Representation based on population

The Senate -- Representation is the same regardless of population (always two).
\subsubsection{The Debate Over Executive Selection}
Areas of Agreement:
\begin{itemize}
    \item One-person executive
    \item President should not be popularly elected.
\end{itemize}
\subsubsection{Electoral College (originally)}
Each state legislature would choose a number of electors equal to the number of its representatives in Congress.

Each elector would vote for two people.

The candidate winning a majority would become president. The person with the next greatest number of votes would become VP.

If no candidate won a majority, the House of Representatives would choose a President, with each state casting one vote.

\subsubsection{Electoral College(today)}
Filters the popular vote in a manner that gives disproportionate weight to less populous states.

Each party selects a slate of candidates. Theoretical possibility of ``faithless electors''.

538: number of electors for state = \# of representatives + \# of senators (always 2).

The Presidential candidate and Vice-Presidential candidate run on the same ticket.

Winner-take-all system in \textit{most} states.

\subsection{Basic Principles of the Constitution}
\begin{description}
    \item[Republicanism] establishes a republic.
    \item[Federalism] division of power between a central and regional governments.
    \item[Seperation of Powers] assigning legislative, judicial, and executive functions to separate government branches.
    \item[Checks and Balances] giving each government branch some scrutiny of control over other branches.
\end{description}

\subsection{Notable Articles}
\begin{enumerate}[label=\Roman*.]
    \item \textbf{Legislative Article}: 18 enumerated powers; Necessary and Proper (Elastic) clause
    \item \textbf{Executive Article}: Presidential duties/powers, qualifications, term, etc.
    \item \textbf{Judicial Article}: Intentionally vague, establishes Supreme Court; no mention of Judicial Review
\end{enumerate}
\underline{\textbf{Others}}

\subsection{Constitutional Change}

%TODO Constitutional Change: The Formal Amendment Process Slide

\subsubsection{Interpretation by the Courts}
\paragraph{Marbury v. Madison (1803)}
provided the basis for the exercise of \textbf{judicial review} in the United States.
\paragraph{Enduring Issue}
\textit{How} should judges interpret the Constitution?
\begin{itemize}
    \item Searching for the original intent of the framers.
    \item Interpreting them in light of the demands of the modern society.
\end{itemize}

\subsection{Trying It All Again}
Framers laid out a \underline{broad} framework of government and were vague or even silent on many issues, allowing for interpretation that can adjust to modern circumstances.

Articles of Confederation leaned too much toward \textbf{freedom} at the expense of \textbf{order}

The current U.S. Constitution strikes a ``judicious balance'' between \textbf{order} and \textbf{freedom}
\begin{itemize}
    \item It paid virtually no attention to \textbf{equality}.
\end{itemize}