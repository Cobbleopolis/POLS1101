\begin{description}
    \item[State] a set of public organizations whose rules are enforced through its command of the means of violence.
    \item[Government] the group of leaders directing the state.
\end{description}
\section{Basic Purposes of Government}
\begin{description}
    \item[Maintaining Order] \hfill
    \begin{description}
        \item[Narrower Meaning of Order] Preserving life and protecting property.
        \item[Social Order] Established patterns of authority in society and traditional modes of behaviors.
    \end{description}
    \item[Providing Public Goods] Benefits and services that are available to everyone.
    \item[Promoting Equality]
    \item[Hobbs] The Leviathan
    \item[Lochte] Believes the government should give more than just security (comparative to Hobbs) and that the government should be limited.
\end{description}
\subsection{Conceptual Framework for Analyzing Government}
\subsubsection{Functional Objectives}
\begin{itemize}
    \item Freedom
    \item Order
    \item Equality
\end{itemize}
\subsubsection{Process}
\begin{itemize}
    \item Majoritarian
    \item Pluralist
\end{itemize}
\subsection{Conceptual Framework for Analyzing Government in the United States}
\begin{itemize}
    \item Maintaining Order
    \item Protecting Freedom
    \begin{itemize}
        \item Freedom \textit{of} \ldots
        \item Freedom \textit{from} \ldots
    \end{itemize}
    \item Promoting Equality
    \begin{description}
        \item[Political Equality] Each citizen has only one vote.
        \item[Social/Economic Eqality] Equality of opportunity v. outcome.
    \end{description}
\end{itemize}
\subsection{Ideology of the Scope of Government}
\subsubsection{Totalitarianism}
The belief that government should have unlimited power.
\begin{itemize}
    \item Object to produce a ``perfect'' society.
\end{itemize}
\subsubsection{Socialism}
An economic theory that lies on a spectrum.
\begin{description}
    \item[Communism] Seeks common ownership of th means of production and property in general.
    \item[Democratic-Socialism] Seeks a democratic-welfare state, through it affirms political rights (e.g., rights to vote and free expression)
    \item[Christian Socialism] Socialism \textit{can} align with various ideologies
    \item Etc., etc.
\end{description}

Having a totalitarian government does not require a socialist economy but they do tend to coincidence.

\subsubsection{Capitalism}
Embraces free enterprise
Private businesses operating for profit, without government regulations.
\begin{description}
    \item[Classical Economics] Operation of th free market is similar to the process of natural selection.
    
    The \textit{narrow} pursuit of profits the \textit{broad} interests of society.
\end{description}

\subsection{Libertarianism}
Political ideology that opposes all government action except what is necessary to protect life and property (\textit{laissez faire})

Both socially- and economically-liberal.

\subsection{Questioning the Liberal-Conservative Continuum}
\begin{itemize}
    \item Only a minority think ideologically
    \item Many do not understand the continuum and select ``moderate'' by default
    \item No longer useful
    \begin{itemize}
        \item Many ``liberals'' no longer favor government activism in general.
        \item Many ``conservatives'' no longer oppose it in principle.
    \end{itemize}
    \item An alternative dimensional framework may be useful\ldots
\end{itemize}

\subsection{If Not Ideology, Then What?}
\subsubsection{Political Knowledge}
\begin{itemize}
    \item Half or more of the public knows the government's basic institution and procedures.
    \begin{itemize}
        \item But knowledge is not equally distributed.
    \end{itemize}
    \item \textit{Education is the single-most important predictor?}
    \begin{itemize}
        \item No meaningful relationship between education and self-placement on the liberal-conservative continuum.
        \begin{itemize}
            \item Individuals with strong beliefs may be impervious to new information that challenges these beliefs.
        \end{itemize} 
    \end{itemize}
\end{itemize}
\subsubsection{The Self-Interest Principle}
People choose what benefits them personally.
\subsubsection{Heuristics}
\begin{itemize}
    \item Examples
    \begin{itemize}
        \item Voting by party label.
        \item Going by the options of religious authorities.
    \end{itemize}
\end{itemize}

\subsection{Procedural Democratic Theory}
Describes the \textit{procedures} followed in making government policies. Addresses 4 questions:
\begin{enumerate}
    \item \textit{Who} should participate in decision-making? \\ Universal participation
    \item \textit{How much} should each participant's vote count? \\ Political equality
    \item \textit{How many} votes are needed to reach a decision? \\ Majority (sometimes plurality) rule
    \item What happens \textit{after} elections? \\ Responsiveness
\end{enumerate}
\subsubsection{Complication: Direct v. Indirect Democracy}
\begin{description}
    \item[Participatory Democracy] Where citizens, themselves (rather than representatives), practice politics.
    \begin{itemize}
        \item Neighborhood governments
    \end{itemize}
    \item[Representative Democracy] Where citizens elect public officials to practice politics on their behalf.
\end{description}
\subsubsection{Problems with Procedural Theory}
\begin{itemize}
    \item Can clash with minority rights.
\end{itemize}

\subsection{Substantive Democracy}
Focus of the \textit{substance} of government policies. Argues that certain principles must be incorporated into govt policies.
\subsubsection{Problems with Substantive Theory}
\begin{itemize}
    \item Not clear on the \textit{desired} substance
    \begin{itemize}
        \item A government is ``truly'' democratic if it proves \textit{x}, \textit{y}, and/or \textit{z}.
    \end{itemize}
\end{itemize}

\section{Models of Democracy}
\subsection{Majoritarian Model}
Interprets ``government by the people'' to mean government by the \textit{majority} of people.
\begin{itemize}
    \item Popular election of government officials
    \item Referenda
    \item Initiatives
    \item Recalls
\end{itemize}
\subsubsection{Key Assertions}
\begin{itemize}
    \item Citizens can control their government if they have adequate mechanisms for popular participation.
    \item Citizens are politically knowledgeable and capable of making rational decisions.
\end{itemize}

\subsection{Pluralist Model}
Interprets ``government by the people'' to mean government by people operating through interest groups competing for influence in a decentralized political structure.
\begin{description}
    \item[Interest groups] An organized group seeking to influence government policy.
\end{description}
Originated in the recognition of America's limited political knowledge/participation.

\subsection{Elite Theory}
A small group (a power elite) makes most important government decisions. Power derived from wealth and business connections (e.g., Dick Cheney). Thus, the U.S. is only superficially democratic, and is really an oligarchy.